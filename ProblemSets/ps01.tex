\documentclass[11pt]{article} 

% ------- List of packages to import ------------
\usepackage{fullpage}
\usepackage{soul}
\usepackage{url}
\usepackage{pgf}
\usepackage{tikz}
\usetikzlibrary{arrows,automata}
\usepackage[latin1]{inputenc}
\usepackage{clrscode3e} % Documentation here http://www.cs.dartmouth.edu/~thc/clrscode/clrscode3e.pdf
\usepackage{amssymb}
\usepackage{amsmath}
\RequirePackage[amsmath,hyperref,thmmarks]{ntheorem}
\usepackage{hyperref} % This always has to be the LAST package.  

\theoremseparator{.}

% ------- Theorem and related environments --------
\newtheorem{theorem}{Theorem}
\newtheorem{conjecture}[theorem]{Conjecture}
\newtheorem{proposition}[theorem]{Proposition}
\newtheorem{claim}[theorem]{Claim}
\newtheorem{lemma}[theorem]{Lemma}
\newtheorem{corollary}[theorem]{Corollary}
\newtheorem{definition}[theorem]{Definition} 
\newtheorem{problem}[theorem]{Problem}
\newtheorem{observation}[theorem]{Observation}
\newtheorem{fact}[theorem]{Fact}

% ------- Commands specifying theorem style -------

\theoremstyle{nonumberplain}
\theoremheaderfont{\normalfont\itshape}
\theorembodyfont{\normalfont}
\theoremsymbol{\ensuremath{_\blacksquare}}
\newtheorem{proof}{Proof}

% ------- Useful math macros -----------
\newcommand{\NN}{\mathcal{N}} % Natural Numbers
\newcommand{\RR}{\mathcal{R}} % Real Numbers
\newcommand{\ZZ}{\mathcal{Z}} % Integers
\newcommand{\QQ}{\mathcal{Q}} % Rational Numbers

\newcommand{\set}[1]{\ensuremath{\{{#1}\}}} % Set
\newcommand{\bigset}[1]{\ensuremath{\left\{{#1}\right\}}}
\newcommand{\condset}[2]{\ensuremath{\set{{#1}\;|\;{#2}}}} % Conditional set
\newcommand{\nin}{\not\in}
\newcommand{\cross}{\times} % Cartesian product
\newcommand{\ssn}{\subsetneq} % Proper subset
\newcommand{\sse}{\subseteq} % Subset




\begin{document}

%% ^^^^^^^^^ Don't change anything above this line ^^^^^^^^^^^^^^^^



\begin{center}

{\LARGE
\textsc{CSC 250 -- Algorithms -- Problem Set 1}
\bigskip}

\bigskip
{\Large
}


\end{center}

% VVVVVVVV Comment out the tutorial below when submitting VVVVVVVVVV

\section*{\LaTeX{} Tutorial}

You will use \LaTeX{} to typeset your solutions to problems in this
course.  You will be provided with a template \texttt{.tex} file for
each problem set that you should modify to typeset your solutions (in
this case \texttt{ps01.tex}).  Below is a brief introduction to
\LaTeX{}.  You can skip this and get right to the problems if you
already know how to use \LaTeX{}.

\LaTeX{} is a programming language for typesetting documents, and is
widely used to prepare scientific papers and textbooks.  Applications
like Microsoft Word and Google Docs are WYSIWYG editors (What You See
Is What You Get), that is, the editor displays the document as it
would be printed.  WYSIWYG editors for better (or worse) allow you
near complete control over how a document is formatted---leaving the
writer to ensure that everything is consistently formatted and
styled. \LaTeX{} on the other hand is not a WYSIWYG editor, instead
you prepare your document by writing a \LaTeX{} program in
\texttt{.tex} file, and then compile it into a document (usually a PDF
file).\footnote{HTML is similar in this regard.}  The beauty of
\LaTeX{} is that it allows you to specify the overall format and style
of the document, then when you compile your document it consistently
applies the format and style to your content without you have to worry
about the details.  Another benefit of \LaTeX{} is the ease with which
you can write mathematical expressions.

\LaTeX{} is open source and free.  You can download its compiler, many
packages, and various editors on your personal machine if you choose.  
You are welcome to use any tool that compiles \texttt{.tex} into
\texttt{.pdf}.  That said, for ease of use, it is suggested that you
use the cloud-based editor Overleaf (\url{https://www.overleaf.com}).

Below are a number of examples of using \LaTeX{} for typesetting.
\ul{I suggest looking at \texttt{ps01.tex} while reading through the
  examples to see the code that produces them.  Use these examples as
  a basis for typesetting your problem set solutions!}

\subsection*{Text}

Any text you type into your \LaTeX{} file will appear as the text does
in this sentence.  You can use various commands to change how text
appears:

\begin{itemize}
\item Text face: \textbf{bold}, \emph{italics}, \ul{underlined},
  \st{strikethrough}.
\item Text size: {\tiny tiny}, {\footnotesize footnote size}, {\small
  small}, {\normalsize normal}, {\large large}, {\Large Large},
{\LARGE LARGE}, {\Huge Huge}
\end{itemize}

\subsection*{Math}

\LaTeX{} makes typesetting math easy!  
\begin{itemize}
\item It can produce Greek symbols,
$\alpha, \beta, \delta, \epsilon, \varepsilon, \gamma, \lambda,
\sigma, \pi, \phi, \iota, \omega, \ldots$ and capital variants,
$\Gamma, \Lambda, \Sigma, \ldots$ 
\item Subscripting and superscripting are
also easy: $2^2, 2^{2^2},2^{2^{2^2}}, w_i,w_{i_j},w_{i_{j_k}},q_0^A$ $\left(\frac{0}{1}\right)^p\left(\frac{1}{0}\right)^p$.
\item There are many built-in operators: $+, -, \cdot, /, \sum, \prod, =,
\neq, \equiv, \le, \ge, \cross, \circ, \rightarrow, \leftarrow,
\leftrightarrow, \Rightarrow, \Leftarrow, \Leftrightarrow, \in, \nin,
\sse, \ssn, \cup, \cap$.  
\item You can define sets: $\NN =
\set{1,2,3,\ldots}$, $\text{EVEN} = \condset{n \in \NN}{n \equiv 0
  \mod 2}$.  
\end{itemize}
Inline math is produced by in-closing text within dollar
signs \texttt{\$}.  You can make mathematical statements appear on
their own lines by using double dollar signs \texttt{\$\$} around
them:
$$2,3,5,7,11,13,17,19,\ldots$$ Here's a reference for math commands:
\url{https://en.wikibooks.org/wiki/LaTeX/Mathematics}


\subsection*{Environments}

An \emph{environment} is \LaTeX{}'s term for a formatting style that
can be applied to a region of content.  For example there are
environments for definitions, theorems, lemmas, claims, and proofs (a
complete list of the available math environments can be found at the
top of \texttt{ps01.tex}).

\begin{definition}
\label{def:prime}
A number $n \in \NN$ is \emph{prime} iff its only divisors are itself
and 1.
\end{definition}

\begin{theorem}
There are infinitely many prime numbers.
\end{theorem}

\begin{proof}
Suppose there are only finitely many prime numbers $1 < p_1 < p_2 <
\ldots < p_n$.  Consider the number $q = 1 + \prod_{i = 1}^n p_i$.
Recalling Definition~\ref{def:prime}\footnote{You can make references
  to things you've written by using the \texttt{label} and
  \texttt{ref} commands.}, observe that $q$ is not divisible by any of
the primes $p_1,p_2,\ldots,p_n$ because $q \equiv 1 \mod p_i$ for all
$i = 1,\ldots,n$.  Therefore $q$ is prime.  This is a contradiction
because $q \neq p_i$ for all $i = 1,\ldots,n$.
% Note you cannot leave a blank line here or it will won't put the
% closing \blacksquare.
\end{proof}

You can create (nested) bulleted lists using the \texttt{itemize}
environment as above in the math section.  You can use the
\texttt{enumerate} environment to create numbered lists, as is done in
the optional exercises section below.

\subsection*{Algorithm}

Our textbook provides a \LaTeX{} package called \texttt{clrscode3e}
for typeseting algorithms.  To use this you'll need to download
\texttt{clrscode3e.sty} off of Nexus and put it in the same folder as
your \texttt{ps01.tex} file.  The algorithm you'll be analyzing can be
found in Problem 1.



\subsection*{\LaTeX{} Tips}

\begin{enumerate}
\item \LaTeX{} compile error messages can be arcane though they
  usually give you a line number.  Check that line.  If the error is
  complete nonsense it is likely you have misplaced a
  \texttt{$\backslash$, \{, \},} or \texttt{\$} somewhere before that
  point.  Comment out lines using the percent character \texttt{\%} to
  isolate the error.
\item Don't reinvent the wheel!  Start from the example code that I've
  provided with this template.  
\item You do not need to understand too much \LaTeX{} to be able to
  produce professional-looking solutions for this class.  If you get
  stuck on something: talk with your peers, search the Internet
  (\url{http://tex.stackexchange.com/} is a useful resource), or talk
  with me!
\item \LaTeX{} has \emph{many} more features than I have time to describe, I
  encourage you to explore them on your own.  One particularly useful
  feature is creating user-defined macros with \texttt{newcommand}.
\item Solve a problem completely \emph{before} writing it up in
  \LaTeX{} (otherwise, you may find yourself with two problems).
\end{enumerate}

\pagebreak


\section*{Required Problems}

\noindent Solve the following problems and typeset your solutions
within this document.  Before compiling this document for submission
comment (or delete) out this paragraph and the above \LaTeX{}
tutorial.  To hand in your solution electronically submit
\texttt{ps01.pdf} to Gradescope.

% ^^^^^^^^^^^ Comment out the above when submitting ^^^^^^^^^^^^^^^^^^

\section*{Problem 1 (60\%)}

Bubblesort is a popular, but inefficient, sorting algorithm.  It works
by repeatedly swapping adjacent elements that are out of order.

\begin{codebox}
\Procname{$\proc{Bubblesort}(A)$}
\li \For $i \gets 1$ \To $\attrib{A}{length} - 1$ \Do
\li     \For $j \gets \attrib{A}{length}$ \Downto $i + 1$ \Do
\li         \If $A[j] < A[j-1]$ \Do
\li             exchange $A[j]$ with $A[j-1]$.
            \End
        \End
    \End
\end{codebox}

\begin{enumerate} 
\item Let $A'$ denote the output of $\proc{Bubblesort}(A)$ and $n =
  \attrib{A}{length}$.  To prove that $\proc{Bubblesort}$ sorts, we
  need to prove that
  \begin{enumerate}
  \item it terminates on all inputs,
  \item its output satisfies the property
    \begin{equation}
      A'[1] \le A'[2] \le \cdots \le A'[n], \text{ and}\label{eqn:ordered}
    \end{equation}
  \item \texttt{XXX}.
  \end{enumerate}
  What is the missing property \texttt{XXX}?  In \emph{one} sentence,
  argue why this additional property holds for \proc{Bubblesort}.
  
\item State precisely a loop invariant for the inner \For loop in
  lines 2--4.  (Hint: Consider what happens to $A[1..(i-1)]$ and
  $A[i..n]$.)

\item State precisely a loop invariant for the outer \For loop in
  lines 1--4 that will allow you to prove~\eqref{eqn:ordered}.  (Hint:
  Think about the insertion sort loop invariant proof from class.)

\item Prove the inner loop invariant you stated holds.  Make sure to
  argue the three properties initialization, maintenance and
  termination.
  
\item Prove the outer loop invariant you stated holds to conclude that
  \proc{Bubblesort} sorts.  Note that because you proved the inner
  loop invariant in the previous part, if you can establish the
  initialization condition of the inner loop invariant you can
  immediately conclude its termination property. 

\item What is the worst-case running time of bubblesort?  How does it
  compare to the running time of insertion sort?
\end{enumerate}


\section*{Problem 2 (30\%)}

Prove the following lemma.

\begin{lemma}
For any natural number $a$ and real number $b > 0$,
$$(b \cdot n + a! \cdot \sqrt{n})^b = \Theta(n^b).$$
\end{lemma}

\begin{proof}

\end{proof}



\section*{Self Reflection (10\%)}

For your final draft, read my posted solution and then respond to the following prompts:
\begin{enumerate}

\item Is your solution correct?  Give a concrete accounting in
  comparison with my posted solution.  (Note: Your solution doesn't
  need to match mine to be correct.)
\item What did you do well and poorly? Explain briefly.
\item What material did you struggle with? Explain briefly.
\item What elements of writing are you going to focus on improving in future problem sets?
  
\end{enumerate}

%% VVVVVVVVVVV Don't change anything below this line VVVVVVVVVVVVVVV

\end{document}
